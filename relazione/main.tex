\documentclass{article}
\usepackage{graphicx} % Required for inserting images
\usepackage{caption}
\usepackage{float}
\usepackage[T1]{fontenc} % Enable guillemets and better font encoding
\usepackage[table]{xcolor} % Enable \rowcolor for tables


\title{\textbf{Relazione Progetto di Tecnologie Web}}
\date{}


\newcommand{\componenti}{
    & Enrico Bianchi \\
    & Davide Martinelli \\
    & Marco Cola \\
    & Riccardo Valerio \\
}
\renewcommand{\contentsname}{Indice}


\begin{document}
\maketitle

\begin{figure}[H]
    \centering
    \includegraphics[width=0.4\textwidth]{logo_unipd.png}
    \label{fig:logo_unipd}
\end{figure}

\vspace{0.5cm}

\begin{center}
\begin{tabular}{r|l}
    \textbf{Componenti} \componenti 
\end{tabular}

\vspace{1cm}
\rule{\linewidth}{0.2mm}

\vspace{0.3cm}

\textbf{Informazioni sul sito} \\
\textbf{Indirizzo sito web:} tecweb.studenti.math.unipd.it/damartin \\
\textbf{Email referente del gruppo: }davide.martinelli.1@studenti.unipd.it

\vspace{0.3cm}
\rule{\linewidth}{0.2mm}

\vspace{0.5cm}
\textbf{Accessi predisposti al sito}
\begin{table}[h]
    \centering
    \begin{tabular}{|p{5cm}|p{5cm}|}
        \hline
        \rowcolor{gray!30}
        Nome utente & Password \\
        \hline
        admin & admin   \\
        \hline
        user & user   \\
        \hline
    \end{tabular}
\end{table}

\end{center}

\newpage

\tableofcontents        %indice

\newpage

\section{Introduzione}
La \textbf{Biblioteca Luzzatti} è una biblioteca universitaria situata in Via Luzzatti 8 a Padova, a pochi passi dall'aula \textbf{LUM250}.  
Il sito web dedicato nasce con l'obiettivo di:

\begin{itemize}
    \item offrire un portale intuitivo e accessibile per la consultazione del catalogo dei volumi disponibili;
    \item semplificare la gestione del patrimonio librario (inserimento, aggiornamento e rimozione dei titoli) tramite un'area riservata allo staff bibliotecario;
    \item garantire un'esperienza di navigazione uniforme su dispositivi desktop e mobile.
\end{itemize}

Il presente documento illustra le scelte progettuali, tecnologiche e organizzative adottate nello sviluppo del sito.


\newpage

\section{Analisi dei requisiti}
Il team ha raccolto i requisiti attraverso colloqui con i bibliotecari, questionari anonimi somministrati nel campus, e l'analisi comparativa di portali accademici già esistenti.  
Particolare attenzione è stata posta all'accessibilità, ai tempi di risposta del catalogo e alla facilità di manutenzione.


\subsection{Tipi di utenti}
\begin{itemize}
    \item \textbf{Visitatore (utente anonimo)} - Può navigare liberamente il catalogo, visualizzare la scheda dettagliata dei volumi, consultare gli orari di apertura e le news sulla vita della biblioteca.  
    \item \textbf{Utente registrato} - Dopo autenticazione dispone, oltre alle funzioni del visitatore, di:
          \begin{itemize}
              \item creazione di una \emph{wishlist} personale;
              \item inserimento di recensioni e valutazioni dei libri;
              \item accesso alla cronologia dei volumi nella wishlist e delle recensioni scritte.
          \end{itemize}
    \item \textbf{Amministratore / Bibliotecario} - Accede a una dashboard riservata per:
          \begin{itemize}
              \item operazioni di \emph{CRUD} (Create-Read-Update-Delete) sui libri presenti nel catalogo;
              \item moderazione delle recensioni;
          \end{itemize}
\end{itemize}


\subsection{Funzionalità}
\begin{itemize}
    \item \textbf{Ricerca avanzata} full-text per titolo e autore;
    \item \textbf{Scheda libro arricchita} (Titolo, dati editoriali, recensioni, trama);
    \item \textbf{Wishlist e recensioni} gestibili dall'utente;
    \item \textbf{Area riservata} con profilo utente e cronologia prestiti;
    \item \textbf{Dashboard amministrativa} per la gestione del catalogo e della moderazione;
    \item \textbf{Layout di stampa};
    \item \textbf{Accessibilità e responsive design} AA WCAG 2.2, con navigazione coerente su desktop e mobile.
\end{itemize}


\section{Progettazione}
Lo schema organizzativo adottato è a \emph{task}, con una chiara suddivisione in quattro macro-sezioni intuitive:

\begin{itemize}
    \item \textbf{Home} - È la prima pagina visualizzata all'accesso al sito.  
          Presenta una panoramica della Biblioteca Luzzatti: missione, orari di apertura e una sintesi dei principali servizi offerti.
    
    \item \textbf{Chi siamo} - Sezione dedicata ai membri del team bibliotecario e al gruppo di progetto: foto, ruoli, breve biografia e competenze e la mappa per raggiungerci. Contiene inoltre un paragrafo sulla storia della biblioteca.
          
    \item \textbf{Catalogo} - Cuore funzionale del sito: elenco completo e ricercabile dei libri presenti, con filtri per titolo e autore. Ogni volume al suo interno rimanda a una \emph{pagina dettaglio} che include la trama del libro, dati editoriali, recensioni e pulsante «Aggiungi a wishlist».
    
    \item \textbf{Area riservata} - Punto d'ingresso per l'autenticazione:  
          \begin{itemize}
              \item \emph{Utente registrato}: accede al proprio profilo, gestisce wishlist, scrive o modifica recensioni;  
              \item \emph{Admin/Bibliotecario}: entra in una dashboard di back-office per operazioni di CRUD sul catalogo dei volumi.
          \end{itemize}
\end{itemize}


\newpage

\section{Organizzazione e Suddivisione del lavoro}
Il lavoro è stato suddiviso come segue:
\begin{itemize}
    \item \textbf{Davide Martinelli} (Project lead, Accessibility):
            \begin{itemize}
                \item testing di accessibilità;
                \item validazione HTML/CSS, contrast-checker;
                \item supporto layout di stampa;
                \item scrittura della relazione progettuale.
            \end{itemize}
    \item \textbf{Marco Cola} (Front-end):
            \begin{itemize}
                \item realizzazione template HTML5 semantici;
                \item miglioramento stile e struttura CSS;
                \item scrittura della relazione progettuale.
            \end{itemize}
    \item \textbf{Enrico Bianchi} (Back-end \& QA):
            \begin{itemize}
                \item progettazione schema database (PHP);
                \item progettazione interfacciamento al database con php;
                \item scrittura della relazione progettuale.
            \end{itemize}
   \item \textbf{Riccardo Valerio} (JavaScript \& UX):
            \begin{itemize}
                \item script di validazione dati lato client;
                \item implementazione iniziale della struttura e dello stile CSS;
                \item scrittura della relazione progettuale.
            \end{itemize}
\end{itemize}


\newpage

\section{Implementazione Front-End}

\subsection{Struttura delle Pagine (HTML)}
Ogni pagina segue la struttura \texttt{header - nav - main - footer}.  
Sono impiegati tag semantici come \texttt{<section>} per migliorare l'accessibilità e la SEO.  
I template sono generati dinamicamente tramite placeholder sostituiti da PHP.


\subsection{Design e Styling (CSS)}
Il foglio \texttt{style.css} definisce un layout responsivo con Flexbox e CSS Grid. 
Il foglio \texttt{printstyle.css} definisce il layout per la stampa delle pagine. 




\subsection{Dinamismo e Interazione (JavaScript)}
Script \texttt{script.js} gestisce:

\begin{itemize}
    \item 
    \item
    \item 
    \item 
\end{itemize}

Script \texttt{val\_admin\_form.js} gestisce:

\begin{itemize}
    \item 
    \item
    \item 
    \item 
\end{itemize}

Script \texttt{val\_reg.js} gestisce:

\begin{itemize}
    \item 
    \item
    \item 
    \item 
\end{itemize}



\subsection{Gestione della Stampa e dei Media (CSS per Stampa)}
Il foglio \texttt{printstyle.css} rimuove elementi di navigazionee converte i colori.

\newpage


\section{Implementazioni back-end}



\subsection{JavaScript}


\subsection{PHP}

Lato server è stato implementato:

\begin{itemize}
    \item
    \item 
    \item gestione errore 505.
\end{itemize}


\section{Accessibilità}
Il sito mira al livello \textbf{AA} delle WCAG 2.2.

\begin{itemize}
    \item Tasti \emph{skip link} posizionati prima della navigazione.
    \item tutti i testi superano AAA.
    \item Tutte le immagini di contenuto includono attributo \texttt{alt};
    \item Form con \texttt{label} associati, \texttt{aria} e messaggi descrittivi.
    \item Navigazione completa via tastiera (tabindex naturale).
\end{itemize}


\subsection{Validazione}
\begin{itemize}
    \item W3C Markup Validator - N errori, N warning (falsi positivi).
    \item W3C CSS Validator - N errori.
\end{itemize}

\end{document}