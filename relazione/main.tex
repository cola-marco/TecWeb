\documentclass{article}
\usepackage{graphicx} % Required for inserting images
\usepackage{caption}
\usepackage{float}
\usepackage[T1]{fontenc} % Enable guillemets and better font encoding
\usepackage[table]{xcolor} % Enable \rowcolor for tables


\title{\textbf{Relazione Progetto di Tecnologie Web}}
\date{}


\newcommand{\componenti}{
    & Enrico Bianchi - 2040978 - Enrico.bianchi.4@studenti.unipd.it\\
    & Davide Martinelli - 2077679 - Davide.martinelli.1@studenti.unipd.it \\
    & Marco Cola - 2079237 - Marco.cola.1@studenti.unipd.it\\
    & Riccardo Valerio - 2075517 -Riccardo.valerio@studenti.unipd.it\\
}
\renewcommand{\contentsname}{Indice}


\begin{document}
\maketitle

\begin{figure}[H]
    \centering
    \includegraphics[width=0.4\textwidth]{logo_unipd.png}
    \label{fig:logo_unipd}
\end{figure}

\vspace{0.5cm}

\begin{center}
\begin{tabular}{r|l}
    \textbf{Componenti} \componenti 
\end{tabular}

\vspace{1cm}
\rule{\linewidth}{0.2mm}

\vspace{0.3cm}

\textbf{Informazioni sul sito} \\
\textbf{Indirizzo sito web:} tecweb.studenti.math.unipd.it/damartin \\
\textbf{Email referente del gruppo: }davide.martinelli.1@studenti.unipd.it

\vspace{0.3cm}
\rule{\linewidth}{0.2mm}

\vspace{0.5cm}
\textbf{Accessi predisposti al sito}
\begin{table}[h]
    \centering
    \begin{tabular}{|p{5cm}|p{5cm}|}
        \hline
        \rowcolor{gray!30}
        Nome utente & Password \\
        \hline
        admin & admin   \\
        \hline
        user & user   \\
        \hline
    \end{tabular}
\end{table}

\end{center}

\newpage

\tableofcontents        %indice

\newpage

\section{Introduzione}
La \textbf{Biblioteca Luzzatti} è una biblioteca universitaria situata in Via Luzzatti 8 a Padova, a pochi passi dall'aula \textbf{LUM250}.  
Il sito web dedicato nasce con l'obiettivo di:

\begin{itemize}
    \item offrire un portale intuitivo e accessibile per la consultazione del catalogo dei volumi disponibili;
    \item semplificare la gestione del patrimonio librario (inserimento, aggiornamento e rimozione dei titoli) tramite un'area riservata allo staff bibliotecario;
    \item garantire un'esperienza di navigazione uniforme su dispositivi desktop e mobile.
\end{itemize}

Il presente documento illustra le scelte progettuali, tecnologiche e organizzative adottate nello sviluppo del sito.


\newpage

\section{Analisi dei requisiti}
Il team ha raccolto i requisiti attraverso colloqui con i bibliotecari, questionari anonimi somministrati nel campus, e l'analisi comparativa di portali accademici già esistenti.  
Particolare attenzione è stata posta all'accessibilità, ai tempi di risposta del catalogo e alla facilità di manutenzione.


\subsection{Tipi di utenti}
\begin{itemize}
    \item \textbf{Visitatore (utente anonimo)} - Può navigare liberamente il catalogo, visualizzare la scheda dettagliata dei volumi, consultare gli orari di apertura e le news sulla vita della biblioteca.  
    \item \textbf{Utente registrato} - Dopo autenticazione dispone, oltre alle funzioni del visitatore, di:
          \begin{itemize}
              \item creazione di una \emph{wishlist} personale;
              \item inserimento di recensioni e valutazioni dei libri;
              \item accesso alla cronologia dei volumi nella wishlist e delle recensioni scritte.
          \end{itemize}
    \item \textbf{Amministratore / Bibliotecario} - Accede a una dashboard riservata per:
          \begin{itemize}
              \item operazioni di \emph{CRUD} (Create-Read-Update-Delete) sui libri presenti nel catalogo;
              \item moderazione delle recensioni;
          \end{itemize}
\end{itemize}


\subsection{Funzionalità}
\begin{itemize}
    \item \textbf{Ricerca avanzata} full-text per titolo e autore;
    \item \textbf{Scheda libro arricchita} (Titolo, dati editoriali, recensioni, trama);
    \item \textbf{Wishlist e recensioni} gestibili dall'utente;
    \item \textbf{Area riservata} con profilo utente e cronologia prestiti;
    \item \textbf{Dashboard amministrativa} per la gestione del catalogo e della moderazione;
    \item \textbf{Layout di stampa};
    \item \textbf{Accessibilità e responsive design} AA WCAG 2.2, con navigazione coerente su desktop e mobile.
\end{itemize}


\section{Progettazione}
Lo schema organizzativo adottato è a \emph{task}, con una chiara suddivisione in quattro macro-sezioni intuitive:

\begin{itemize}
    \item \textbf{Home} - È la prima pagina visualizzata all'accesso al sito.  
          Presenta una panoramica della Biblioteca Luzzatti: missione, orari di apertura e una sintesi dei principali servizi offerti.
    
    \item \textbf{Chi siamo} - Sezione dedicata ai membri del team bibliotecario e al gruppo di progetto: foto, ruoli, breve biografia e competenze e la mappa per raggiungerci. Contiene inoltre un paragrafo sulla storia della biblioteca.
          
    \item \textbf{Catalogo} - Cuore funzionale del sito: elenco completo e ricercabile dei libri presenti, con filtri per titolo e autore. Ogni volume al suo interno rimanda a una \emph{pagina dettaglio} che include la trama del libro, dati editoriali, recensioni e pulsante «Aggiungi a wishlist».
    
    \item \textbf{Area riservata} - Punto d'ingresso per l'autenticazione:  
          \begin{itemize}
              \item \emph{Utente registrato}: accede al proprio profilo, gestisce wishlist, scrive o modifica recensioni;  
              \item \emph{Admin/Bibliotecario}: entra in una dashboard di back-office per operazioni di CRUD sul catalogo dei volumi.
          \end{itemize}
\end{itemize}


\newpage

\section{Organizzazione e Suddivisione del lavoro}
Il lavoro è stato suddiviso come segue:
\begin{itemize}
    \item \textbf{Davide Martinelli} (Project lead, Accessibility):
            \begin{itemize}
                \item testing di accessibilità;
                \item validazione HTML/CSS, contrast-checker;
                \item supporto layout di stampa;
                \item scrittura della relazione progettuale.
            \end{itemize}
    \item \textbf{Marco Cola} (Front-end):
            \begin{itemize}
                \item realizzazione template HTML5 semantici;
                \item miglioramento stile e struttura CSS;
                \item scrittura della relazione progettuale.
            \end{itemize}
    \item \textbf{Enrico Bianchi} (Back-end \& QA):
            \begin{itemize}
                \item progettazione schema database (PHP);
                \item progettazione interfacciamento al database con php;
                \item scrittura della relazione progettuale.
            \end{itemize}
   \item \textbf{Riccardo Valerio} (JavaScript \& UX):
            \begin{itemize}
                \item script di validazione dati lato client;
                \item implementazione iniziale della struttura e dello stile CSS;
                \item scrittura della relazione progettuale.
            \end{itemize}
\end{itemize}


\newpage

\section{Implementazione Front-End}

\subsection{Struttura delle Pagine (HTML)}

Durante la realizzazione del sito web della \textbf{Biblioteca Luzzatti}, è stata definita una struttura HTML modulare e semanticamente coerente, progettata per essere accessibile, riutilizzabile e facilmente estendibile.  
Tutte le pagine seguono un’impostazione comune basata su elementi ricorrenti, ordinati secondo il seguente schema:

\begin{center}
\texttt{\textless header\textgreater{} - \textless nav\textgreater{} - \textless main\textgreater{} - \textless footer\textgreater{}}
\end{center}

La testata del sito, definita in \texttt{header.html}, include un titolo racchiuso in un tag \texttt{<h1>} nascosto visivamente ma accessibile agli screen reader, e un menu di navigazione responsive che si trasforma in pulsante \texttt{hamburger} su schermi piccoli.  
Sono inoltre presenti metadati dinamici per migliorare l’indicizzazione nei motori di ricerca, e un link per saltare direttamente al contenuto principale, pensato per favorire la navigazione assistita.

Il menu varia in base al ruolo dell’utente: se anonimo, vengono visualizzate le voci “Accedi” e “Registrati”; una volta autenticato, compare la voce “Il mio profilo”, e se l’utente ha privilegi di amministratore, è disponibile anche la sezione “Admin”. La voce corrispondente alla pagina attiva non è cliccabile, per evitare ambiguità di navigazione.

A eccezione della homepage e della pagina di errore, ogni pagina include un breadcrumb implementato con \texttt{<nav aria-label="breadcrumb">}, che mostra la posizione corrente nel sito e consente di tornare agevolmente alle sezioni superiori.

Il contenuto principale è contenuto nel tag \texttt{<main id="contenuto">}, con sezioni organizzate gerarchicamente tramite \texttt{<section>} e \texttt{<article>}, titoli ben strutturati, e — dove necessario — contenuti dinamici inseriti tramite placeholder come \texttt{\#\#\#TITOLO\#\#\#}, \texttt{\#\#\#LISTA\#\#\#}, sostituiti a runtime dal codice PHP.

Il piè di pagina, incluso da \texttt{footer.html}, riporta le informazioni di contatto della biblioteca, i link ai canali social e una tabella oraria degli orari di apertura, realizzata con tag semantici come \texttt{<time>} e attributi \texttt{aria-describedby} per facilitarne la lettura automatica.  
In fondo ad ogni pagina è presente anche un link “Torna su”, identificato da \texttt{id="torna-primo"}.

\medskip

Ogni pagina del sito è costruita per soddisfare una funzione specifica:

\begin{itemize}
  \item \texttt{index.html} funge da homepage, presentando due sezioni principali che descrivono l’identità della biblioteca e i servizi offerti, corredate da immagini descrittive e ben etichettate.
  
  \item \texttt{chi-siamo.html} ospita diverse sezioni informative: una breve storia della biblioteca, la missione, i membri del team con link ai profili GitHub, e una mappa integrata per localizzarla fisicamente.

  \item \texttt{catalogo.html} mostra dinamicamente l’elenco dei libri, organizzato orizzontalmente con una lista \texttt{<ul class="slider">}. È presente un modulo di ricerca semantico che invia i dati a \texttt{search.php}.

  \item \texttt{libro.html} fornisce i dettagli del libro selezionato, impaginati in due colonne (immagine + dettagli testuali). Comprende anche un’area recensioni, accessibile e ben strutturata, con votazioni in stelle.

  \item \texttt{area-riservata.html} mostra la wishlist e il profilo personale dell’utente autenticato, divisi in due colonne affiancate, con contenuti caricati dinamicamente tramite placeholder come \texttt{\#\#\#LISTA-WISHLIST\#\#\#}.

  \item \texttt{admin.html} e \texttt{areariservatatest.html} permettono la gestione dei volumi da parte dell’amministratore. I dati sono mostrati in una tabella accessibile, con intestazioni semantiche e righe caricate tramite \texttt{\#\#\#BODY\_TABELLA\#\#\#}.

  \item \texttt{login.html} e \texttt{register.html} contengono form costruiti con attenzione all’accessibilità: campi con label associate, messaggi di errore gestiti via \texttt{aria-describedby} e aggiornati in modo dinamico con \texttt{aria-live="polite"}.

  \item \texttt{505.html} è la pagina di errore del server: semplice, chiara, e contenente un link di ritorno alla homepage.
\end{itemize}

Tutti i file HTML rispettano la sintassi XML e sono stati validati tramite il \textit{W3C Markup Validator}, risultando conformi agli standard.

\medskip

La gestione dei contenuti dinamici avviene tramite un sistema di sostituzione lato server, che consente di caricare nelle pagine solo i dati necessari, separando in modo netto la struttura (HTML), la logica (PHP) e la presentazione (CSS). Questa scelta aumenta la modularità e riduce la duplicazione del codice.

\bigskip

\noindent
\textbf{Accessibilità e ottimizzazione SEO}

Il codice HTML è stato scritto seguendo i principi dell’accessibilità digitale:

\begin{itemize}
  \item utilizzo di landmark semantici (\texttt{<header>}, \texttt{<nav>}, \texttt{<main>}, \texttt{<footer>});
  \item uso corretto di attributi \texttt{aria-*}, testi alternativi nelle immagini, e struttura gerarchica dei titoli;
  \item navigazione da tastiera completa, con link “salta al contenuto” e focus visibile;
  \item URL semanticamente significativi e contenuti ben marcati per migliorare l’indicizzazione da parte dei motori di ricerca.
\end{itemize}

\noindent
Nel complesso, la struttura HTML del sito rispetta pienamente gli obiettivi del progetto: offrire un’interfaccia accessibile, organizzata in maniera semantica, facilmente espandibile e perfettamente compatibile con tutti i dispositivi.


\subsection{Design e Styling (CSS)}

L’aspetto grafico del sito è stato interamente sviluppato a mano tramite un unico foglio di stile principale, \texttt{style.css}, scritto in CSS3. L’obiettivo era ottenere un design coerente, moderno, leggibile e completamente responsivo, mantenendo una forte attenzione all’accessibilità e alla compatibilità cross-browser.

Lo stile è organizzato per sezioni tematiche e commentate, ciascuna delle quali si occupa di una parte specifica dell’interfaccia utente: intestazione e menu, breadcrumb, struttura delle pagine, area utente, schede dei libri, sistema di recensioni, form, footer. La struttura del CSS è stata pensata per facilitare manutenzione, estendibilità e riutilizzo.

\medskip

\noindent
Alla base della coerenza visiva del sito vi è l’uso di variabili CSS, definite nel selettore \texttt{:root}. Variabili come \texttt{--headerbgcolor}, \texttt{--footerbgcolor}, \texttt{--linkcolor}, \texttt{--txtcolor} e \texttt{--buttoncolor} permettono di gestire facilmente l’identità cromatica del sito e garantiscono uno stile uniforme.

Dal punto di vista del layout, il sito utilizza una combinazione di Flexbox e CSS Grid per distribuire i contenuti. Le sezioni principali come la \texttt{.user-area} (che mostra wishlist e profilo) o i contenuti dinamici (es. la lista dei libri) sono disposti in griglie flessibili che si adattano alla larghezza dello schermo. L’intera impaginazione è progettata per essere responsiva: grazie alle media query, il menu di navigazione passa da orizzontale a \texttt{hamburger} sotto i 1000px, e molte sezioni verticali si impilano automaticamente sui dispositivi mobili.

Le card dei libri, sono definite attraverso la classe \texttt{.card} e presentano un’immagine della copertina affiancata ai dati testuali. Sono progettate per essere facilmente scalabili e leggibili su qualsiasi dispositivo, con un effetto di zoom all’hover per migliorare la fruizione su desktop.

Le recensioni degli utenti sono gestite graficamente tramite \texttt{.card-recensione}, che imposta margini, padding, sfondi, e bordi arrotondati per rendere ogni opinione facilmente distinguibile, ordinata e leggibile. Anche i form di login, registrazione e inserimento recensioni seguono uno stile uniforme, con etichette ben visibili, focus evidenziato e gestione visiva degli errori tramite classi come \texttt{.error-msg}.

Per quanto riguarda l’accessibilità visiva, ogni colore è stato scelto per garantire un contrasto minimo di livello AA secondo le WCAG 2.2. I link, i pulsanti e gli elementi interattivi hanno uno stato \texttt{:hover} e \texttt{:focus} visibile, che facilita l’uso da parte di utenti che navigano tramite tastiera o screen reader.

\medskip




\subsection{Dinamismo e Interazione (JavaScript)}
Script \texttt{script.js} gestisce:

\begin{itemize}
    \item 
    \item
    \item 
    \item 
\end{itemize}

Script \texttt{val\_admin\_form.js} gestisce:

\begin{itemize}
    \item 
    \item
    \item 
    \item 
\end{itemize}

Script \texttt{val\_reg.js} gestisce:

\begin{itemize}
    \item 
    \item
    \item 
    \item 
\end{itemize}



\subsection{Gestione della Stampa e dei Media (CSS per Stampa)}
Il foglio \texttt{printstyle.css} rimuove elementi di navigazionee converte i colori.

\newpage


\section{Implementazioni back-end}



\subsection{JavaScript}


\subsection{PHP}

Lato server è stato implementato:

\begin{itemize}
    \item
    \item 
    \item gestione errore 505.
\end{itemize}


\section{Accessibilità}
Il sito mira al livello \textbf{AA} delle WCAG 2.2.

\begin{itemize}
    \item Tasti \emph{skip link} posizionati prima della navigazione.
    \item tutti i testi superano AAA.
    \item Tutte le immagini di contenuto includono attributo \texttt{alt};
    \item Form con \texttt{label} associati, \texttt{aria} e messaggi descrittivi.
    \item Navigazione completa via tastiera (tabindex naturale).
\end{itemize}


\subsection{Validazione}
\begin{itemize}
    \item W3C Markup Validator - N errori, N warning (falsi positivi).
    \item W3C CSS Validator - N errori.
\end{itemize}

\end{document}